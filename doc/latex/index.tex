{\bfseries Note\+: This is N\+O\+T the user\textquotesingle{}s guide! For help getting started, please, visit the {\itshape Getting Started} page of the Wiki, here\+: \href{https://github.com/capn-freako/ibisami/wiki/Getting-Started}{\tt https\+://github.\+com/capn-\/freako/ibisami/wiki/\+Getting-\/\+Started}}

{\bfseries Note\+: This documentation was generated, using \href{http://www.stack.nl/~dimitri/doxygen/index.html}{\tt Doxygen}.}

\subsection*{Introduction}

{\itshape ibisami} is a public domain package of C++ code, and associated support files, intended to provide a common public code base for the generic portion of an I\+B\+I\+S-\/\+A\+M\+I model.

The ibisami code base is hosted by Git\+Hub and is available, here\+: \href{https://github.com/capn-freako/ibisami}{\tt https\+://github.\+com/capn-\/freako/ibisami}

The original commit was posted by \href{mailto:capn.freako@gmail.com}{\tt David Banas} on April 29, 2015.

The code is released under the B\+S\+D3 license, specifically, in order to avoid any concerns of \char`\"{}virality\char`\"{}. That is, it is intended that this code be usable by all, and that no one making modifications to it is under any obligation to share those modifications with anyone for any reason. This holds, regardless of whether or not those modifications are used for commercial purposes.

M\+A\+K\+I\+N\+G I\+M\+P\+R\+O\+V\+E\+M\+E\+N\+T\+S T\+O T\+H\+I\+S C\+O\+D\+E, U\+S\+I\+N\+G T\+H\+O\+S\+E I\+M\+P\+R\+O\+V\+E\+M\+E\+N\+T\+S F\+O\+R C\+O\+M\+M\+E\+R\+C\+I\+A\+L P\+U\+R\+P\+O\+S\+E\+S, A\+N\+D K\+E\+E\+P\+I\+N\+G T\+H\+O\+S\+E I\+M\+P\+R\+O\+V\+E\+M\+E\+N\+T\+S E\+N\+T\+I\+R\+E\+L\+Y T\+O Y\+O\+U\+R\+S\+E\+L\+F I\+S T\+O\+T\+A\+L\+L\+Y A\+C\+C\+E\+P\+T\+A\+B\+L\+E.

(Of course, we hope you\textquotesingle{}ll share with us, but if you don\textquotesingle{}t that\textquotesingle{}s our problem, not yours. ;-\/) )

\subsection*{Contributing to ibisami development}

If you would like to help out in maintaining/improving this code\+:
\begin{DoxyItemize}
\item That is great! Thank you!
\item Please, fork your own version of the Git\+Hub repository, O\+N G\+I\+T\+H\+U\+B. \href{https://help.github.com/articles/fork-a-repo/}{\tt Instructions}
\item Please, D\+O N\+O\+T clone my {\itshape ibisami} repository to your working machine. (You won\textquotesingle{}t be able to push your improvements up to my Git\+Hub repository, if you do this.)
\item When you have something you\textquotesingle{}d like me to include, send me a {\itshape pull request}. \href{https://help.github.com/articles/creating-a-pull-request/}{\tt Instructions} I will pull your changes into a separate branch, which I keep expressly for this purpose, test your code, and, if all goes well, incorporate your changes into a new release of {\itshape ibisami} without delay.
\end{DoxyItemize}

{\bfseries Note\+: It may seem a little clunky to do things this way, but it makes collaborative project management M\+U\+C\+H easier. Thanks, in advance, for your cooperation.} 